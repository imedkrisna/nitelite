% Pandoc/Quarto template for den-paper extension
\documentclass[
  manuscript=article,
  layout=publish,
  year=2026,
  volume=3,
]{den}

% DOI

% Conference (for proceedings/poster)

% Dates
\published{9 February 2026}

% Editor and reviewers

% Strip abstract fields from bibliography entries (prevents LuaLaTeX parsing issues)

% Title
\title{Forecasting Indonesian National and Provincial GDP using
nighttime light index}

% Authors and affiliations
\author{Krisna Gupta \orcid{0000-0001-8695-0514}}
\affiliation{National Economic Council, Republic of Indonesia}
\email{krisna@dewanekonomi.go.id}
\author{Timothy Kinmekita Ginting}
\affiliation{National Economic Council, Republic of Indonesia}
\email{timothy.ginting@dewanekonomi.go.id}
\author{Meizahra Afidatie}
\affiliation{National Economic Council, Republic of Indonesia}
\email{meizahra@dewanekonomi.go.id}

% Keywords
\keywords{Night Light; Growth Forecasting; BPS}

% Abbreviations

% Pandoc/Quarto compatibility macros
\providecommand{\tightlist}{%
  \setlength{\itemsep}{0pt}\setlength{\parskip}{0pt}}
\providecommand{\citeproc}[2]{#1}
\providecommand{\citeproctext}{}

% Support for graphics
\RequirePackage{graphicx}

% Support for ding symbols (checkmarks, etc.)
\RequirePackage{pifont}

% Support for table column width calculations (required by Quarto-generated tables)
\RequirePackage{calc}

% Pandoc image handling - required for newer Pandoc/Quarto versions
% Constrains images to text width to prevent overflow
\makeatletter
\providecommand{\pandocbounded}[1]{%
  \sbox0{#1}%
  \ifdim\wd0>\linewidth
    \resizebox{\linewidth}{!}{#1}%
  \else
    #1%
  \fi
}
\makeatother

% CSL references support
\newlength{\cslhangindent}
\setlength{\cslhangindent}{1.5em}
\newlength{\csllabelwidth}
\setlength{\csllabelwidth}{3em}
\makeatletter
\newcommand{\CSLNoBibLabel}{\renewcommand{\@biblabel}[1]{}}
\makeatother
\newenvironment{CSLReferences}[2]
 {\begin{list}{}%
  {\setlength{\leftmargin}{0pt}
   \setlength{\itemindent}{0pt}
   \setlength{\itemsep}{#2\baselineskip}
   \setlength{\parsep}{0pt}
   \setlength{\topsep}{0pt}
   \setlength{\partopsep}{0pt}
   \ifodd #1
     \setlength{\leftmargin}{\cslhangindent}
     \setlength{\itemindent}{-\cslhangindent}
   \fi}
  \CSLNoBibLabel}
 {\end{list}}

% Header includes
\makeatletter
\@ifpackageloaded{tcolorbox}{}{\usepackage[skins,breakable]{tcolorbox}}
\@ifpackageloaded{fontawesome5}{}{\usepackage{fontawesome5}}
\definecolor{quarto-callout-color}{HTML}{909090}
\definecolor{quarto-callout-note-color}{HTML}{0758E5}
\definecolor{quarto-callout-important-color}{HTML}{CC1914}
\definecolor{quarto-callout-warning-color}{HTML}{EB9113}
\definecolor{quarto-callout-tip-color}{HTML}{00A047}
\definecolor{quarto-callout-caution-color}{HTML}{FC5300}
\definecolor{quarto-callout-color-frame}{HTML}{acacac}
\definecolor{quarto-callout-note-color-frame}{HTML}{4582ec}
\definecolor{quarto-callout-important-color-frame}{HTML}{d9534f}
\definecolor{quarto-callout-warning-color-frame}{HTML}{f0ad4e}
\definecolor{quarto-callout-tip-color-frame}{HTML}{02b875}
\definecolor{quarto-callout-caution-color-frame}{HTML}{fd7e14}
\makeatother
\makeatletter
\@ifpackageloaded{caption}{}{\usepackage{caption}}
\AtBeginDocument{%
\ifdefined\contentsname
  \renewcommand*\contentsname{Table of contents}
\else
  \newcommand\contentsname{Table of contents}
\fi
\ifdefined\listfigurename
  \renewcommand*\listfigurename{List of Figures}
\else
  \newcommand\listfigurename{List of Figures}
\fi
\ifdefined\listtablename
  \renewcommand*\listtablename{List of Tables}
\else
  \newcommand\listtablename{List of Tables}
\fi
\ifdefined\figurename
  \renewcommand*\figurename{Figure}
\else
  \newcommand\figurename{Figure}
\fi
\ifdefined\tablename
  \renewcommand*\tablename{Table}
\else
  \newcommand\tablename{Table}
\fi
}
\@ifpackageloaded{float}{}{\usepackage{float}}
\floatstyle{ruled}
\@ifundefined{c@chapter}{\newfloat{codelisting}{h}{lop}}{\newfloat{codelisting}{h}{lop}[chapter]}
\floatname{codelisting}{Listing}
\newcommand*\listoflistings{\listof{codelisting}{List of Listings}}
\makeatother
\makeatletter
\makeatother
\makeatletter
\@ifpackageloaded{caption}{}{\usepackage{caption}}
\@ifpackageloaded{subcaption}{}{\usepackage{subcaption}}
\makeatother

% Keep internal links clickable without PDF border underlines.
\hypersetup{pdfborder={0 0 0}}

% Source note environment for figure attributions (right-aligned, compact)
\newenvironment{source}{%
  \vspace{-1em}%
  \begin{flushright}\scriptsize\itshape
}{%
  \end{flushright}%
}
\newcommand{\sumber}[1]{%
  \vspace{-1em}%
  \begin{flushright}\scriptsize\itshape
  Sumber: #1
  \end{flushright}%
}

% Allow LaTeX more flexibility in line breaking to prevent overflow
% (needed for Indonesian/non-English text on narrow B5 paper)
\tolerance=1000
\emergencystretch=1.5em

\begin{document}

\begin{abstract}
Economic growth is a central macroeconomic indicator that shapes the
decisions of both governments and private enterprises. Therefore,
tracking economic growth in higher frequency would benefit decision
makers. One way to verify the official growth number is to use relevant
leading indicators for economic growth that are independent from the
statistical agency. In this paper, we use the Indonesian nighttime light
index sourced from Blackmarble to fit historical GDP of Indonesia. Two
forms of analysis are conducted: national-level using Autoregressive
Distributed Lag (ARDL) and provincial level using panel data regression.
The nighttime light index shows a consistent significant correlation to
GDP growth across various specifications. However, the magnitude of the
correlation is small, suggesting that nighttime light index alone cannot
capture the full variation of GDP growth. We also find indicative
evidence that the scarring effect post COVID-19 pandemic hurts long term
economic growth by 2\%. The ARDL specification with scarring effect show
the best forecasting fit and is reasonably able to forecast GDP growth
out-of-sample.
\end{abstract}

\begin{tcolorbox}[enhanced jigsaw, opacitybacktitle=0.6, titlerule=0mm, bottomrule=.15mm, colback=white, leftrule=.75mm, left=2mm, rightrule=.15mm, opacityback=0, coltitle=black, toprule=.15mm, bottomtitle=1mm, breakable, colframe=quarto-callout-note-color-frame, title=\textcolor{quarto-callout-note-color}{\faInfo}\hspace{0.5em}{Note}, toptitle=1mm, arc=.35mm, colbacktitle=quarto-callout-note-color!10!white]

DEN Working Papers describe research in progress by the author(s) and
are published to elicit comments and to encourage debate. The views
expressed in DEN Working Papers are those of the author(s) and do not
necessarily represent the views of the Dewan Ekonomi Nasional.

\end{tcolorbox}

\section{Introduction}\label{introduction}

GDP and economic growth are arguably the most significant sources of
data for the government. Economic growth rate is used as an anchor for
various other indicators. It forms the foundation for critical modeling
and analysis used by both governments and private investors to make
economic decisions and implement policy measures. Because GDP often
serves as a crucial performance indicator for the government, there is
an incentive to overestimate the growth number (Martínez 2022). It is
therefore essential to develop alternative methods to validate and
evaluate economic growth data.

One such method lies in the use of nighttime lights as a proxy to
nowcast economic growth. The use of satellite imagery, particularly in
the form of nighttime lights, has increased in relevance over the last
20 years. Technology has developed to allow for the detection of signals
at night coming from common artificial light sources such as
streetlights, buildings, and vehicles. This data can then be used to
measure human activity, a critical component of economic growth.
Nighttime lights growth serves as a good predictor of economic growth at
the national and sub-national levels (Henderson, Storeygard, and Weil
2012; Bickenbach et al. 2016; Martínez 2022). Henderson, Storeygard, and
Weil (2012) shows how nighttime lights data are able to serve as a
better predictor of economic growth than various indicators and proxies
in other countries. The fact that nighttime lights data is procured from
NASA as an open source ensures full transparency. The data is readily
available without any pre-processing or involvement from third parties,
meaning it is immune to the fluctuations in perceived credibility that
are associated with statistical agencies. The independence from
statistical agencies is an important condition that positions nighttime
lights well as a leading indicator for GDP growth (Enders 2014).

In this paper, we utilize a raster of monthly nighttime lights data from
Indonesia provided by NASA's Black Marble project (Stefanini Vicente and
Marty 2023). We then average the data into quarterly, mirroring the GDP
data from BPS. We then fit nighttime lights index on real GDP growth
using OLS and Autoregressive Distributed Lag (ARDL) models. The ARDL
model showed the most promising fit. Importantly, we find evidence of a
potential structural break post COVID-19. Additionally, we provide a
provincial analysis to provide cross-sectional variance. We utilise OLS,
provincial fixed effect and two-way fixed effect regression.

This paper aims to test whether nighttime light can be a sole predictor
of GDP without adding other variables. We find that nighttime light
index provides a consistent significant correlation to GDP growth across
various specifications. However, the magnitude of the correlation is
small, suggesting that nighttime light index alone cannot capture the
full variation of GDP growth. However, We also find indicative evidence
that the scarring effect post COVID-19 pandemic (Pangestu and Armstrong
2025) hurts long term economic growth by 2\%. The ARDL specification
with scarring effect show the best forecasting fit and is reasonably
able to forecast GDP growth out-of-sample.

The paper is organised as follows. We discuss the nighttime lights data
collection process and exploratory data analysis section two. The
methodology development is covered in section three. Section four
discusses the model results, followed by a conclusion in section five.

\section{Data Collection and
Processing}\label{data-collection-and-processing}

NASA Black Marble (Stefanini Vicente and Marty 2023) is a a daily
calibrated, corrected, and validated product suite, curated such that
nighttime lights data can be used effectively for scientific
observations. The product suite takes full advantage of the capabilities
of the Visible Infrared Imaging Radiometer Suite (VIIRS) instrument,
which is a component of the Suomi National Polar-orbiting Partnership
(NPP) satellite. The instrument consists of 22 spectral bands from the
ultra-violet to the infrared, of which the day night band (DNB) in
particular is used to observe nighttime lights. The DNB is
ultra-sensitive, and can detect very dim light that is several times
fainter than daylight. The band covers 0.5--0.9 µm wavelengths (visible
green to near-infrared), which is exactly the range of light emitted by
common artificial sources like streetlights, buildings, vehicles, and
even fishing boats.

While the analysis of nighttime lights has become more popular over the
last two decades, a surprisingly few number of studies employ the use of
data from VIIRS (Gibson, Olivia, and Boe-Gibson 2020). The new nighttime
lights data offers a sharper resolution and higher frequency compared to
the previous generation of nighttime lights data. Black Marble's
standard science removes cloud-contaminated pixels and and corrects for
atmospheric, terrain, vegetation, snow, lunar, and stray light effects
on the VIIRS instrument.

The data collection process was performed using the Black Marble Python
package developed by the World Bank (Stefanini Vicente and Marty 2023).
After mapping and defining Indonesia's coordinates as the region of
interest, we were able to use the \texttt{blackmarblepy} package to
access NASA Black Marble as a xarray dataset. NASA's Black Marble suite
offers daily, monthly, and yearly global nighttime lights data. Rasters
were able to be created at all three frequency levels. Each xarray
dataset contains a nighttime lights tile that is gap-filled and
corrected, with a resolution of 500m. Critically, each dataset also
contains a main variable representing radiance, a numerical measure of
the amount of light energy emitted or reflected from a surface per unit
area in a given direction, expressed in watts per square meter per
steradian \((W \cdot m^{-2} \cdot sr^{-1})\). It is this measure that
allows for nighttime lights to be compared and used as a proxy for GDP
growth.

\begin{figure}[htbp!]

\centering{

\pandocbounded{\includegraphics[keepaspectratio]{fig/nitelite.png}}

}

\caption{\label{fig-1}Annual Nighttime Lights in Indonesia, 2023}

\end{figure}%

Figure~\ref{fig-1} The figure is a visualization of the yearly raster
for nighttime lights in Indonesia in 2023. There is a stark contrast
between the nighttime lights activity in Java compared to other islands,
which is reflective of significant gaps in various socioeconomic
indicators between Java and the rest of Indonesia. The stark difference
in economic activity between Java and the rest of Indonesia is
well-documented in literature, and is a consequence of the landscape and
soil of the island facilitating stronger agricultural yields and
population growth.

Black Marble data can also be extracted for multiple time periods. The
function will return a raster stack, where each raster band corresponds
to a different date. The following code snippet provides examples of
getting data across multiple days, for the month of May 2024 in
Indonesia. We define a date range using pd.date\_range.

\begin{figure}[htbp!]

\centering{

\pandocbounded{\includegraphics[keepaspectratio]{fig/file_show.png}}

}

\caption{\label{fig-may}Daily Average Nighttime Light Radiance in
Indonesia, May 2024}

\end{figure}%

Here we can see the fluctuations that exist within a given month,
fluctuations that may be difficult to pinpoint from monthly or yearly
rasters. One advantage of the flexibility of nighttime lights data is
the ability to process it to suit the needs of any kind of time series
analysis. In this instance, to facilitate the goal of making a proper
comparison between nighttime lights and GDP, both series needed to be
expressed on the same unit level. In Indonesia, GDP growth is typically
reported in quarterly year-on-year terms. To align the nighttime lights
data with this format, multiple steps were needed. First, monthly
rasters were extracted from January 2012 to December 2024, covering the
full period of available Black Marble nighttime lights data. The data
was then saved as a .zip file. The radiance values were also extracted
and saved as a separate .csv file.

With the radiance values extracted in a monthly form, the next steps
involved transforming the data into quarterly year-on-year terms.
Nighttime lights data was aggregated into quarterly terms. The data was
then lagged and shifted 1 year back, from which the year-on-year change
was able to be calculated.

GDP data was straightforward to collect due to the data being readily
available from the BPS website (BPS 2025). Quarterly real GDP is used
for the purposes of this study. The GDP series includes data from
Q1-2010 to Q2-2025, but we will use Q1-2012 as our starting point in
line with the availability of nighttime lights data from NASA Black
Marble. We also collets the provincial quarterly real GDP for the
provincial-level regression.

\begin{figure}[htbp!]

\centering{

\pandocbounded{\includegraphics[keepaspectratio]{fig/figQ.png}}

}

\caption{\label{fig-2}Indonesian economic growth and night light growth}

\end{figure}%

Figure~\ref{fig-2} shows our two main variables. The left panel is the
national real GDP sourced from BPS (2025) while the right panel is the
calculated night lights we gathered via Black Marble python package.
Both seems to follow similar trend. However, night light index doesn't
show significant drop during the COVID time, unlike GDP. Additionally,
the nightlight index looks to be more stationary before 2022, and then
show a positive trend. From the visual inspection, both series seems to
be non-stationary.

\section{Methodology}\label{methodology}

For the national level, we do not have any cross-sectional variation.
Therefore, techniques that utilise cross-sectional mean cannot be
exploited. Multivariate time series techniques, thus, should be the
appropriate method.

First, we try a simple OLS: \[
g_t=\alpha_0+\alpha_1 ntli_t + \epsilon_t
\]

where \(g_t\) is the log of quarterly GDP at time \(t\), \(ntli_t\) is
the log quarterly night light index at time \(t\), \(\alpha_0\) is the
constant term, \(\alpha_1\) is the coefficient of night light index, and
\(\varepsilon_t\) is the error term. We then examine the error term and
see if there is a bias in the residuals. It is reasonable to find
autocorrelation in the residual with two time series data. We then use
ARDL to take into account the autocorrelation (Enders 2014). We use lags
determined by AIC.

\[
g_t=\beta_0+\sum_{p=1}^{P}\beta_i g_{t-i}+\sum_{q=0}^{Q}\beta_j ntli_{t-j}+\epsilon_t
\]

In addtion to the above specification, we also test with quarter dummy,
Covid dummy (2020-2022) and scarring dummy (2020 onwards).

For the regional level analysis allows for a cross-sectional dimension
to be added to the time series data. This allows us to use panel data
techniques. We can use provincial fixed effect and two-way fixed effect
(TWFE) models.

In addition to the cross-sectional techniques, we employ panel
distributed lag techniques to account for the dynamic nature of the
relationship between nighttime lights and regional GDP as in the
national level. Namely, we follow the approach of Mahraddika (2019) by
using Dynamic Fixed Effects (DFE), Mean Group (MG) and Pooled Mean Group
(PMG) estimators Pesaran, Shin, and Smith (1999).

\section{Estimation Results}\label{estimation-results}

\subsection{National-level analysis}\label{national-level-analysis}

\begin{figure}[htbp!]

\centering{

\pandocbounded{\includegraphics[keepaspectratio]{fig/Qols.png}}

}

\caption{\label{fig-3}OLS fit and the residuals}

\end{figure}%

\begin{longtable}[]{@{}ll@{}}
\caption{OLS Regression Results for log real quarterly
GDP}\label{tbl-ols}\tabularnewline
\toprule\noalign{}
Variable & Coefficient \\
\midrule\noalign{}
\endfirsthead
\toprule\noalign{}
Variable & Coefficient \\
\midrule\noalign{}
\endhead
\bottomrule\noalign{}
\endlastfoot
const & 15.4878*** \\
& (0.0561) \\
ntlg & 0.5401*** \\
& (0.0407) \\
Observations & 55 \\
R-squared & 0.7683 \\
Adj. R-squared & 0.7639 \\
F-statistic & 175.76 \\
\end{longtable}

Figure~\ref{fig-3} shows the OLS fit and the residuals while
Table~\ref{tbl-ols} show the regression result. The nighttime light show
a strong correlation to the quarterly GDP. 1\% increase in nighttime
light index corresponds with 0.57\% increase in GDP. However,
Figure~\ref{fig-3} shows a clear bias on the residuals. The model
overpredicts GDP pre-2017 and underpredicts GDP post 2017. This suggest
that OLS is not the proper method to model these two series.

\begin{figure}[htbp!]

\centering{

\pandocbounded{\includegraphics[keepaspectratio]{fig/ARDLQ.png}}

}

\caption{\label{fig-4}Quarterly real GDP, ARDL}

\end{figure}%

Figure~\ref{fig-4} shows the ARDL approach. We ran six different
specifications. The panel (a) shows a baseline model with only log GDP
and log nighttime light. Panel (b) adds COVID dummies, which consist of
values equaling 1 for the year 2020-2022. Panel (c) adds scarring
dummies, equaling 1 for all observations beginning from 2020. Panel (d)
adds quarterly dummies. Panel (e) combines COVID dummies with quarterly
dummies, while panel (f) use scarring dummies and quarterly dummies. The
solid line represents the actual GDP data while the dashed line
illustrates the predicted GDP.

We can see from the Figure~\ref{fig-4} that all predicted values follow
the actual GDP very well. However, the GDP value predicted by model with
scarring, that is, both panel (c) and (f), looks to have the smallest
discrepancy with the actual data.

\begin{longtable}[]{@{}
  >{\raggedright\arraybackslash}p{(\linewidth - 12\tabcolsep) * \real{0.1591}}
  >{\raggedright\arraybackslash}p{(\linewidth - 12\tabcolsep) * \real{0.1364}}
  >{\raggedright\arraybackslash}p{(\linewidth - 12\tabcolsep) * \real{0.1364}}
  >{\raggedright\arraybackslash}p{(\linewidth - 12\tabcolsep) * \real{0.1364}}
  >{\raggedright\arraybackslash}p{(\linewidth - 12\tabcolsep) * \real{0.1591}}
  >{\raggedright\arraybackslash}p{(\linewidth - 12\tabcolsep) * \real{0.1364}}
  >{\raggedright\arraybackslash}p{(\linewidth - 12\tabcolsep) * \real{0.1364}}@{}}
\caption{ARDL Regression Results}\label{tbl-ardl}\tabularnewline
\toprule\noalign{}
\begin{minipage}[b]{\linewidth}\raggedright
Variable
\end{minipage} & \begin{minipage}[b]{\linewidth}\raggedright
Baseline
\end{minipage} & \begin{minipage}[b]{\linewidth}\raggedright
+Covid
\end{minipage} & \begin{minipage}[b]{\linewidth}\raggedright
+Scar
\end{minipage} & \begin{minipage}[b]{\linewidth}\raggedright
+Quarterly
\end{minipage} & \begin{minipage}[b]{\linewidth}\raggedright
+Q+C
\end{minipage} & \begin{minipage}[b]{\linewidth}\raggedright
+Q+S
\end{minipage} \\
\midrule\noalign{}
\endfirsthead
\toprule\noalign{}
\begin{minipage}[b]{\linewidth}\raggedright
Variable
\end{minipage} & \begin{minipage}[b]{\linewidth}\raggedright
Baseline
\end{minipage} & \begin{minipage}[b]{\linewidth}\raggedright
+Covid
\end{minipage} & \begin{minipage}[b]{\linewidth}\raggedright
+Scar
\end{minipage} & \begin{minipage}[b]{\linewidth}\raggedright
+Quarterly
\end{minipage} & \begin{minipage}[b]{\linewidth}\raggedright
+Q+C
\end{minipage} & \begin{minipage}[b]{\linewidth}\raggedright
+Q+S
\end{minipage} \\
\midrule\noalign{}
\endhead
\bottomrule\noalign{}
\endlastfoot
const & 2.2942 & 2.9864 & 49.8166*** & 2.5612* & 2.8221* & 37.7501*** \\
& (1.5716) & (1.7920) & (5.2152) & (1.3806) & (1.4232) & (4.9545) \\
trend & 0.0008 & 0.0020 & 0.0424*** & 0.0013 & 0.0020* & 0.0321*** \\
& (0.0011) & (0.0014) & (0.0045) & (0.0010) & (0.0011) & (0.0042) \\
g.L1 & 0.5393*** & 0.5947*** & -0.7533*** & 0.6644*** & 0.6247*** &
-0.6215*** \\
& (0.1316) & (0.1331) & (0.1196) & (0.1254) & (0.1061) & (0.1025) \\
g.L2 & -0.0804 & -0.4507*** & -0.9372*** & 0.1776 & 0.0812 &
-0.4841*** \\
& (0.1638) & (0.1561) & (0.0865) & (0.1660) & (0.1338) & (0.1249) \\
g.L3 & 0.0036 & 0.2549 & -0.7755*** & -0.2861* & -0.1873 & -0.5542*** \\
& (0.1682) & (0.1627) & (0.1224) & (0.1691) & (0.1371) & (0.1103) \\
g.L4 & 0.3857*** & 0.3967*** & 0.0147 & 0.2708** & 0.2861*** & 0.0454 \\
& (0.1352) & (0.1234) & (0.0766) & (0.1275) & (0.1020) & (0.0631) \\
covid.L0 & & -0.0003 & & & -0.0031 & \\
& & (0.0098) & & & (0.0082) & \\
covid.L1 & & -0.0482*** & & & -0.0401*** & \\
& & (0.0129) & & & (0.0108) & \\
covid.L2 & & 0.0370*** & & & 0.0318*** & \\
& & (0.0136) & & & (0.0086) & \\
covid.L3 & & -0.0272* & & & & \\
& & (0.0147) & & & & \\
covid.L4 & & 0.0262** & & & & \\
& & (0.0113) & & & & \\
ntlg.L0 & 0.0855*** & 0.0350 & 0.0119* & 0.0571** & 0.0312* &
0.0115** \\
& (0.0229) & (0.0227) & (0.0059) & (0.0212) & (0.0183) & (0.0048) \\
ntlg.L1 & 0.0033 & -0.0350 & & -0.0275 & -0.0372** & \\
& (0.0245) & (0.0209) & & (0.0207) & (0.0176) & \\
ntlg.L2 & -0.0460** & & & & & \\
& (0.0224) & & & & & \\
q1.L0 & & & & & & -0.0108*** \\
& & & & & & (0.0023) \\
q2.L0 & & & & 0.0299*** & 0.0273*** & \\
& & & & (0.0077) & (0.0062) & \\
q3.L0 & & & & 0.0293*** & 0.0278*** & 0.0079*** \\
& & & & (0.0078) & (0.0062) & (0.0025) \\
scar.L0 & & & -0.0201*** & & & -0.0204*** \\
& & & (0.0043) & & & (0.0035) \\
scar.L1 & & & -0.0965*** & & & -0.0937*** \\
& & & (0.0064) & & & (0.0053) \\
scar.L2 & & & -0.0621*** & & & -0.0400*** \\
& & & (0.0127) & & & (0.0114) \\
scar.L3 & & & -0.0679*** & & & -0.0308** \\
& & & (0.0090) & & & (0.0115) \\
scar.L4 & & & -0.0342*** & & & -0.0276*** \\
& & & (0.0110) & & & (0.0094) \\
Observations & 51 & 51 & 51 & 51 & 51 & 51 \\
AIC & -275.94 & -291.52 & -408.79 & -287.61 & -309.81 & -428.67 \\
BIC & -256.62 & -264.48 & -383.68 & -266.36 & -282.76 & -399.69 \\
\end{longtable}

Table~\ref{tbl-ardl} shows the estimated parameters of the 6 models.
Indeed, models with scarring show the lowest AIC and BIC, which suggests
these specifications are the most robust compared to the other
specifications without scarring. The scarring effect leads to a 2\%
decrease in quarterly GDP. Importantly, the current nighttime light
index is significant for all but the COVID-dummy specification. The base
model shows the strongest correlation, where a 1\% increase in nighttime
light index correlates to a 0.0855\% increase in GDP. For the scarring
specification, the coefficients are 0.0119 and 0.0115 respectively. That
is to say that a 1\% increase in nighttime light index correlates with a
0.0119\% increase in GDP.

\begin{figure}[htbp!]

\centering{

\pandocbounded{\includegraphics[keepaspectratio]{fig/ARDL_train_test_forecast.png}}

}

\caption{\label{fig-5}Economic growth prediction with ARDL}

\end{figure}%

To test whether the model can forecast GDP well, we divide the data into
a training set consisting all observation prior to 2024, and testing set
consists of the remaining data. The result is illustrated in
Figure~\ref{fig-5}. The solid line represents the actual GDP, the dashed
line represents the predicted GDP using the training set, and the dotted
line represents the forecast using the testing set. We can see from the
Figure~\ref{fig-5} that the model with scarring dummies can predict with
a better fit and smaller error rate.

Notably, the GDP prediction is less sensitive to nighttime lights
fluctuations compared to the scarring dummy and the GDP lags. In fact,
the nighttime light index becomes less useful as the training size
decreases (which is reflected in the AIC and BIC). In the model with
scarring dummy, the nighttime light coefficient at time 0 is only
significant in 10\% with or without quarterly dummies.

Indeed, the past GDP values are more important, with the AR1 coefficient
is -0.7 and is significant at 0.1\% level. More importantly, scarring
effect shows a 0.1\% level significance at all 4 lags. Scarring dummy at
time 0 has a coefficient of -0.02, suggestiong about 2\% lower in GDP
level persistently after the pandemic.

\subsection{Regional-level analysis}\label{regional-level-analysis}

In parallel with the national level specification, we also extended the
baseline specification by including dummy variables for the Covid-19
period (2020-2022) and the post-pandemic or the scarring effect period
(2022-2025) to assess whether the relationship between provincial GDP
and nighttime lights changed during and after the pandemic. We conduct
the analysis using a range of econometric models. The analysis begin
with pooled OLS as a baseline and further estimation using Fixed Effects
(FE), Two-Way Fixed Effects (TWFE), and Dynamic Fixed Effects (DFE).
Across all model specifications, the relationship between GDRP and
nighttime lights remains consistent, indicating that nighttime light
data provides a useful proxy for GDRP.

\begin{figure}[htbp!]

\centering{

\pandocbounded{\includegraphics[keepaspectratio]{lala/fig/scatter_ntl.png}}

}

\caption{\label{fig-l1}Nighttime-Lights vs.~Regional GDP, log}

\end{figure}%

Figure~\ref{fig-l1} is a scatterplot of nighttime lights against
regional GDP for 34 provinces in Indonesia from 2014Q1 to 2024Q4. Java
exhibits a stark contrast compared to the rest of Indonesia. NIghttime
light index show a substantially higher intensity along with its RGDP
(Regional GDP) compared to other island groups. The data points for Java
in blue color are clustered toward the higher end of both axes,
indicating higher levels of economic activity and luminosity relative to
other regions. Conversely, though the linear trend between nighttime
light and provincial GDP are still visible, provinces outside Java show
greater dispersion. This implies more variation in the relationship
between light intensity and GDP possibly due to differences in economic
structure or spatial distribution of economic activities. In regions
outside Java, the economic structure is dominated by agriculture,
plantation activities, and mining industries. While these sectors
contribute significantly to regional output, they produce comparatively
low levels of nighttime lights.

\begin{longtable}[]{@{}lrrrr@{}}
\caption{Summary statistics of nighttime lights (ln) and GDRP (ln) at
quarterly frequency, 2014-2024}\label{tbl-sumstat}\tabularnewline
\toprule\noalign{}
& Mean & Std. Dev. & Min & Max \\
\midrule\noalign{}
\endfirsthead
\toprule\noalign{}
& Mean & Std. Dev. & Min & Max \\
\midrule\noalign{}
\endhead
\bottomrule\noalign{}
\endlastfoot
\textbf{Sumatera (10 provinces)} & & & & \\
Nighttime lights (ln) & -1.44 & 0.69 & -3.35 & 0.23 \\
GDRP (ln) & 10.59 & 0.80 & 8.82 & 12.01 \\
\textbf{Java (6 provinces)} & & & & \\
Nighttime lights (ln) & 1.03 & 1.17 & -1.36 & 3.71 \\
GDRP (ln) & 12.04 & 1.02 & 9.65 & 13.24 \\
\textbf{Nusa Tenggara (3 provinces)} & & & & \\
Nighttime lights (ln) & -1.43 & 1.33 & -4.00 & 0.70 \\
GDRP (ln) & 10.01 & 0.37 & 9.23 & 10.70 \\
\textbf{Kalimantan (5 provinces)} & & & & \\
Nighttime lights (ln) & -2.30 & 0.69 & -4.14 & -0.75 \\
GDRP (ln) & 10.35 & 0.74 & 9.02 & 11.90 \\
\textbf{Sulawesi (6 provinces)} & & & & \\
Nighttime lights (ln) & -2.15 & 0.67 & -4.01 & -0.93 \\
GDRP (ln) & 9.77 & 0.87 & 8.22 & 11.55 \\
\textbf{Maluku (2 provinces)} & & & & \\
Nighttime lights (ln) & -2.96 & 0.51 & -4.37 & -1.52 \\
GDRP (ln) & 8.79 & 0.33 & 8.19 & 9.77 \\
\textbf{Papua (2 provinces)} & & & & \\
Nighttime lights (ln) & -3.16 & 0.46 & -4.32 & -2.22 \\
GDRP (ln) & 9.98 & 0.52 & 9.09 & 10.89 \\
\end{longtable}

Table~\ref{tbl-sumstat} shows the summary statistics of the nighttime
light and regional GDP by islands. The table reinforce what is visually
apparent in Figure~\ref{fig-l1}: Java is the most developed island in
Indonesia, with the highest average nighttime light intensity and
regional GDP. It is remains to be seen if the positive relationships are
different among provinces.

The varying degree of light-GDP relationship across islands (and, by
extension, across provinces) suggests that using nighttime lights as a
proxy for economic activity may require adjustments based on regional
characteristics. For instance, provinces with economies heavily reliant
on non-luminous sectors may not exhibit a strong correlation between
light intensity and GDP. This highlights the importance of considering
local economic structures when utilizing nighttime lights data for
economic analysis. Indeed, potentially looking at national-level only
might mask these regional differences.

\begin{table}

\caption{\label{tbl-regols}Panel Regression Results for log provincial GDP}

\centering{

\centering
\begin{tabular}{lccccccccc} \hline
\toprule
 & \multicolumn{3}{c}{OLS} & \multicolumn{3}{c}{FE} & \multicolumn{3}{c}{TWFE} \\
\cmidrule(lr){2-4}\cmidrule(lr){5-7}\cmidrule(lr){8-10}


VARIABLES & Base & +Cov & +Scar & Base & +Cov & +Scar & Base & +Cov & +Scar \\ \hline
 & (1) & (2) & (3) & (4) & (5) & (6) & (7) & (8) & (9) \\ \hline
 &  &  &  &  &  &  &  &  &  \\

ln\_ntl & 0.561*** & 0.560*** & 0.559*** & 0.284*** & 0.257*** & 0.189*** & 0.0547*** & 0.0547*** & 0.0547*** \\
 & (0.0124) & (0.0123) & (0.0123) & (0.0327) & (0.0279) & (0.0251) & (0.0184) & (0.0184) & (0.0184) \\
covid &  & 0.110** &  &  & 0.147*** &  &  & 0.420*** &  \\
 &  & (0.0443) &  &  & (0.0128) &  &  & (0.0210) &  \\
scarring &  &  & 0.0814** &  &  & 0.185*** &  &  & 0.458*** \\
 &  &  & (0.0410) &  &  & (0.0145) &  &  & (0.0220) \\
Constant & 11.33*** & 11.30*** & 11.30*** & 10.92*** & 10.84*** & 10.71*** & 10.33*** & 10.33*** & 10.33*** \\
 & (0.0292) & (0.0314) & (0.0329) & (0.0484) & (0.0416) & (0.0380) & (0.0325) & (0.0325) & (0.0325) \\
 &  &  &  &  &  &  &  &  &  \\
Observations & 1,632 & 1,632 & 1,632 & 1,632 & 1,632 & 1,632 & 1,632 & 1,632 & 1,632 \\
R-squared & 0.546 & 0.547 & 0.547 & 0.349 & 0.489 & 0.582 & 0.845 & 0.845 & 0.845 \\
OLS & plain & Covid & Scarring &  &  &  &  &  &  \\
Number of prov &  34 & 34 & 34 & 34 & 34 & 34 & 34 & 34 & 34 \\
\bottomrule
\multicolumn{10}{l}{\scriptsize Standard errors in parentheses. *** $p<0.01$, ** $p<0.05$, * $p<0.1$.} \\
\end{tabular}

}

\end{table}%

Table~\ref{tbl-regols} shows the regional panel regression. As the
baseline (column OLS), we estimate the correlation between nighttime
lights (NTL) and regional GDP using a pooled Ordinary Least Squares
(OLS) model. the next three columns employ a fixed-effects (FE) model to
control for time-invariant provincial characteristics. The last column
show a two-way fixed-effects (TWFE) model that advanced the FE
estimation by incorporating year fixed effects to account for common
shocks affecting all provinces simultaneously. With applied time
fixed-effects, the scarring and COVID dummies are no longer needed.

The regression results show a strong and statistically significant
relationship at the 0.1 percent level between regional GDP and nighttime
light intensity across all static model specifications. Coefficients
from the OLS and fixed-effects models remain positive and stable even
after controlling for COVID-19 and post-pandemic scarring effects. The
overall elasticity pattern between regional GDP and nighttime light
intensity remains consistent over time.

The coefficient on the OLS model corroborates that of the national level
OLS. The FE model show a slightly smaller coefficient, suggesting the
importance of the provincial difference. Controlling for time fixed
effect, the coefficient drops to 0.0649. That is, a 1\% increase in
nighttime light index corresponds to a 0.0547\% increase in regional
GDP.

To account for the possibility that the relationship between GDP and
nighttime lights evolves dynamically rather than instantaneously, we
extend the analysis using a Dynamic Fixed Effects (DFE), Pooled Mean
Group (PMG) and Mean Group (MG) error-correction model Pesaran, Shin,
and Smith (1999). Unlike the static FE and TWFE specifications, the
distributed lag framework allows for short-run fluctuations while
modeling a long-run equilibrium relationship between the variables.
Specifically, it incorporates lagged adjustments so that deviations from
the long-run relationship can gradually converge back to equilibrium. At
the same time, the model retains fixed effects (for DFE case) to control
for unobserved, time-invariant provincial heterogeneity.

The DFE results provide strong evidence of a stable long-run
relationship between the two variables. The error-correction term is
negative and statistically significant across all specifications.
Importantly, the magnitude and significance of the error-correction
coefficient remain robust even after controlling for the post-pandemic
scarring periods. This suggests that, although the pandemic caused
temporary disruptions, it did not fundamentally alter the long-run
linkage between economic activity and nighttime light intensity.

\setlength{\tabcolsep}{3pt}

\begin{longtable}{l*{6}{c}}

\caption{\label{tbl-regdfe}Panel ARDL--ECM Estimates (DFE vs MG vs PMG)}

\tabularnewline

 \\
\toprule
 & \multicolumn{2}{c}{DFE} & \multicolumn{2}{c}{MG} & \multicolumn{2}{c}{PMG} \\
\cmidrule(lr){2-3}\cmidrule(lr){4-5}\cmidrule(lr){6-7}
 & (1) Baseline & (2) +Scarring
 & (3) Baseline & (4) +Scarring
 & (5) Baseline & (6) +Scarring \\
\midrule
\endhead
\midrule
\multicolumn{7}{r}{{Continued on next page}} \\
\endfoot

\endlastfoot
\multicolumn{7}{l}{\textbf{Panel A. Long-Run Relationship}}\\
\addlinespace[3pt]

ln\_ntl
& 0.332*** & 0.589**
& 0.486*** & -0.713
& 0.374    & 0.426 \\
& (0.114)  & (0.283)
& (0.155)  & (1.098)
& (0.379)  & (1.113) \\

scarring
&          & -0.100
&          & -0.0128
&          & -0.0490 \\
&          & (0.186)
&          & (0.121)
&          & (1.113) \\

\addlinespace[6pt]
\midrule
\multicolumn{7}{l}{\textbf{Panel B. Short-Run Dynamics}}\\
\addlinespace[3pt]

ECT
& -0.0374*** & -0.0169*
& -0.0908*** & -0.0395**
& -0.0418    & -0.0457 \\
& (0.00851)  & (0.00939)
& (0.0154)   & (0.0179)
& (0.399)    & (0.719) \\

\addlinespace[4pt]
\multicolumn{7}{l}{Lagged $\Delta \ln(\text{pdrb})$}\\
L1.
& -0.602*** & -0.654***
& -0.620*** & -1.029***
& -0.578*** & -0.931*** \\
& (0.0255)  & (0.0265)
& (0.0349)  & (0.0623)
& (0.0339)  & (0.0637) \\

L2.
& -0.551*** & -0.583***
& -0.597*** & -1.014***
& -0.539*** & -0.908*** \\
& (0.0270)  & (0.0277)
& (0.0332)  & (0.0633)
& (0.0313)  & (0.0644) \\

L3.
& -0.507*** & -0.544***
& -0.546*** & -0.963***
& -0.473*** & -0.860*** \\
& (0.0265)  & (0.0270)
& (0.0400)  & (0.0669)
& (0.0363)  & (0.0678) \\

L4.
& 0.264***  & 0.270***
& 0.199***  & -0.103*
& 0.278***  & -0.0135 \\
& (0.0250)  & (0.0251)
& (0.0306)  & (0.0527)
& (0.0335)  & (0.0552) \\

\addlinespace[4pt]
\multicolumn{7}{l}{Lagged $\Delta \ln(\text{ntl})$}\\
D
& 0.0130** & 0.00275
& -0.00254 & -0.00271
& 0.0229***& -0.00150 \\
& (0.00593)& (0.00526)
& (0.00841)& (0.00910)
& (0.00659)& (0.00414) \\

L1.
& 0.0101*  & 0.000825
& -0.00978 & -0.00671
& 0.00867  & -0.00830 \\
& (0.00576)& (0.00512)
& (0.00821)& (0.0101)
& (0.00590)& (0.00569) \\

L2.
& 0.0170***& 0.00564
& 0.00532  & 0.00270
& 0.0193***& 0.00170 \\
& (0.00530)& (0.00473)
& (0.00768)& (0.00824)
& (0.00614)& (0.00551) \\

L3.
& 0.00425  & -0.00491
& -0.00235 & -0.00299
& 0.00577  & -0.00602 \\
& (0.00439)& (0.00392)
& (0.00628)& (0.00794)
& (0.00524)& (0.00648) \\

L4.
& -0.00271 & -0.00439
& 0.000898 & -0.000289
& 0.00453  & -0.00186 \\
& (0.00327)& (0.00290)
& (0.00454)& (0.00445)
& (0.00427)& (0.00363) \\

\addlinespace[4pt]
\multicolumn{7}{l}{Scarring dynamics (only for +Scarring models)}\\
D
&          & -0.0563***
&          & -0.0560***
&          & -0.0601*** \\
&          & (0.00527)
&          & (0.00491)
&          & (0.00484) \\

L1.
&          & -0.000834
&          & -0.0230***
&          & -0.0187*** \\
&          & (0.00544)
&          & (0.00730)
&          & (0.00560) \\

L2.
&          & -0.0707***
&          & -0.0748***
&          & -0.0752*** \\
&          & (0.00521)
&          & (0.00642)
&          & (0.00571) \\

L3.
&          & -0.0407***
&          & -0.0770***
&          & -0.0720*** \\
&          & (0.00557)
&          & (0.00940)
&          & (0.00799) \\

L4.
&          & 0.0380***
&          & 0.00610
&          & 0.0110* \\
&          & (0.00549)
&          & (0.00607)
&          & (0.00575) \\
\addlinespace[6pt]
\hline
\multicolumn{7}{l}{\scriptsize Standard errors in parentheses. *** $p<0.01$, ** $p<0.05$, * $p<0.1$.} \\

\end{longtable}

Table~\ref{tbl-regdfe} shows the dynamic fixed effects regression
results. The results show a strong and statistically significant
relationship at the 0.1 percent level between regional GDP and nighttime
light intensity across all static model specifications. Coefficients
from the OLS and fixed-effects models remain positive and stable even
after controlling for COVID-19 and post-pandemic scarring effects. The
overall elasticity pattern between regional GDP and nighttime light
intensity remains consistent over time.

The distributed lag results corroborates the national results. That is,
the short-term changes in nighttime lights have a weak implication to
the GDP's convergence to its long-term equilibrium. More importantly,
while the scarring is not significant in the long-term, the small
error-correction term suggesting a slow adjustment to the long-term
equilibrium after shocks.

With the ECT value of -0.0374 in the baseline DFE model, it would take
approximately 27 quarters (or about 6.75 years) for regional GDP to
fully adjust back to its long-run equilibrium after a shock. The results
corroborates Reinhart and Rogoff (2014)'s findings on the slow
adjustment of GDP to its long-term trend around 6-8 years.

However, when we include the scarring dummy, the ECT value decreases to
-0.0169, indicating an even slower adjustment process. In this case, it
would take approximately 59 quarters (or about 14.75 years) for regional
GDP to fully revert to its long-run equilibrium following a shock. This
suggests that the post-pandemic scarring effect has prolonged the time
it takes for regional economies to recover and stabilize after
disruptions.

Recent studies have highlighted the possibility of a hysteresis, or a
long-lasting, potentially permanent impact of a deep crisis Reinhart and
Rogoff (2014). When an economy face a deep crisis, it can have a lasting
impact amid an increase of risk permia which hinders banks from lending
to firms. This creates a feedback loop as firms are unable to invest and
grow, corroding the economic growth potential. Indonesia's experience
from the 1998 crisis provides a historical precedent for such hysteresis
effects, where growth rates never really returned to pre-Asian Financial
Crisis level.

\section{Conclusion}\label{conclusion}

We ran national-level and regional-level regression to inspect the
relationship between nighttime light index and GDP. Overall, we find
that nighttime light index can be used to predict GDP growth well.
However, there are some caveats that need to be highlighted. Analyzing
national-level and regional-level provide us with bi-variate time series
analysis coupled with insights from panel data regression. We find that
the OLS results are significant, large, and provide consistent
estimators when regional cross-sectional variation is added. However,
OLS models provide biased estimators for both the national-level and
regional-level analysis. Once we take into account autocorrelation (from
the ARDL specification) and provincial bias (using fixed effects), the
correlation strength of the nightlight index decreases.

There are two potential key takeaways from the national-level analysis.
First, the scarring effect is critical for forecasting GDP, more than it
might be for other indicators. We can see this especially when we use
quarterly dummies. From Figure~\ref{fig-4}, The panel (c) and (f) show
that quarterly dummies do not significantly add insight, while panel (d)
and (e), panels with quarterly dummies without scarring, show a looser
fit. This finding is consistent with other studies that examine the
potential scarring effect from the COVID pandemic in Indonesia (Pangestu
and Armstrong 2025).

Secondly, the nightlight index does not perform as well as we think. We
find that that the nighttime light index fluctuates in a different way
compared to GDP. This is most apparent during the COVID-19 pandemic,
where GDP drops significantly, while the nighttime light index
experiences only a modest drop.

Various potential reasons can explain this phenomenon. On days where the
satellite view of the Earth is obstructed by clouds, nighttime lights
values are often gap-filled and estimated based on the number of
available clear pixels. This means that the nighttime light index might
not accurately capture the actual emitted light in Indonesia amid cloud
cover, atmospheric conditions, or other factors that affect the quality
of satellite imagery especially during the rainy months of the tropical
region.

Secondly, the Indonesian government has been increasingly implementing
electricity subsidies during hard times. The nighttime lights index is
unable to identify situations involving a heavy downturn where the
government decided to step in to maintain electricity consumption.

The regional level analysis corroborates the national-level findings.
That is, a potential existence of hysteresis, or the scarring effect of
pandemic, is found in the regional-level analysis as well. The DFE base
results confirms the literature speed of convergence, but the scarring
effect reduces the speed of convergence even further. Additionally, the
nightlight index also not as important as we think, as the short-term
dynamics show weak significance.

Nevertheless, the ARDL model shows promise in forecasting national GDP.
The out-of-sample forecast shows that the model can predict GDP well,
especially when we use scarring dummies using a minimal number of
variables. While we have demonstrate that nighttime light index cannot
be used as a sole predictor of GDP, its consistent significance across
various specifications suggest that nighttime light data can be
considered to feed forecasting and nowcasting models.

\section*{Reproducibility Statement}\label{reproducibility-statement}
\addcontentsline{toc}{section}{Reproducibility Statement}

The source code and data to reproduce the analysis in this paper are
available at \url{https://www.github.com/den-econ/nitelite}.

\section*{References}\label{references}
\addcontentsline{toc}{section}{References}

\phantomsection\label{refs}
\begin{CSLReferences}{1}{1}
\bibitem[\citeproctext]{ref-aikman}
Aikman, David, Mathias Drehmann, Mikael Juselius, and Xiaochuan Xing.
2022. \emph{The Scarring Effects of Deep Contractions}. Bank of Finland
Research Discussion Papers 12/2022. Helsinki: Bank of Finland.
\url{https://hdl.handle.net/10419/265328}.

\bibitem[\citeproctext]{ref-nl2}
Bickenbach, Frank, Eckhardt Bode, Peter Nunnenkamp, and Mareike Söder.
2016. {``Night Lights and Regional GDP.''} Journal Article. \emph{Review
of World Economics} 152 (2): 425--447.
https://doi.org/\url{https://doi.org/10.1007/s10290-016-0246-0}.

\bibitem[\citeproctext]{ref-blanchard}
Blanchard, Olivier, Eugenio Cerutti, and Lawrence Summers. 2015.
\emph{Inflation and Activity â€`` Two Explorations and Their Monetary
Policy Implications}. NBER Working Papers 21726. National Bureau of
Economic Research, Inc. \url{https://doi.org/None}.

\bibitem[\citeproctext]{ref-bps}
BPS. 2025. \emph{{[}Seri 2010{]} 4. Laju Pertumbuhan PDB Menurut
Pengeluaran, 2025. Diakses Pada 14 September 2025}.
Https://www.bps.go.id/id/statistics-table/2/MTA4IzI=/-seri-2010--4--laju-pertumbuhan-pdb-menurut-pengeluaran--persen-.html.

\bibitem[\citeproctext]{ref-enders}
Enders, Walter. 2014. \emph{Applied Econometric Time Series}. Wiley.

\bibitem[\citeproctext]{ref-nle}
Gibson, John, Susan Olivia, and Geua Boe-Gibson. 2020. {``Night Lights
in Economics: Sources and Uses.''} Journal Article. \emph{Journal of
Economic Surveys} 34: 955--980.
https://doi.org/\url{https://doi.org/10.1111/joes.12387}.

\bibitem[\citeproctext]{ref-nl}
Henderson, J. Vernon, Adam Storeygard, and David N. Weil. 2012.
{``Measuring Economic Growth from Outer Space.''} \emph{The American
Economic Review} 102 (2): 994--1028.
\url{http://www.jstor.org/stable/23245442}.

\bibitem[\citeproctext]{ref-lucidi}
Lucidi, Francesco Simone, and Willi Semmler. 2023. {``Long-Run Scarring
Effects of Meltdowns in a Small-Scale Nonlinear Quadratic Model.''}
\emph{Journal of Macroeconomics} 75: 103487.
https://doi.org/\url{https://doi.org/10.1016/j.jmacro.2022.103487}.

\bibitem[\citeproctext]{ref-wishnu}
Mahraddika, Wishnu. 2019. {``Does International Reserve Accumulation
Crowd Out Domestic Private Investment?''} Journal Article.
\emph{International Economics} 158: 39--50.
https://doi.org/\url{https://doi.org/10.1016/j.inteco.2019.02.003}.

\bibitem[\citeproctext]{ref-nl3}
Martínez, Luis R. 2022. {``How Much Should We Trust the Dictator's GDP
Growth Estimates?''} Journal Article. \emph{Journal of Political
Economy} 130 (10): 2731--2769. \url{https://doi.org/10.1086/720458}.

\bibitem[\citeproctext]{ref-pangestu}
Pangestu, Mari, and Shiro Armstrong. 2025. {``Is This Time Different?
Indonesia's Response to Uncertainty.''} \emph{Bulletin of Indonesian
Economic Studies} 61 (3): 297--332.
\url{https://doi.org/10.1080/00074918.2025.2588819}.

\bibitem[\citeproctext]{ref-pesaran2}
Pesaran, M. Hashem, Yongcheol Shin, and Ron P. Smith. 1999. {``Pooled
Mean Group Estimation of Dynamic Heterogeneous Panels.''} Journal
Article. \emph{Journal of the American Statistical Association} 94
(446): 621--634. \url{https://doi.org/10.2307/2670182}.

\bibitem[\citeproctext]{ref-pesaran}
Pesaran, M. Hashem, and Ron Smith. 1995. {``Estimating Long-Run
Relationships from Dynamic Heterogeneous Panels.''} Journal Article.
\emph{Journal of Econometrics} 68 (1): 79--113.
https://doi.org/\url{https://doi.org/10.1016/0304-4076(94)01644-F}.

\bibitem[\citeproctext]{ref-rogoff}
Reinhart, Carmen M., and Kenneth S. Rogoff. 2014. {``Recovery from
Financial Crises: Evidence from 100 Episodes.''} \emph{The American
Economic Review} 104 (5): 50--55.
\url{http://www.jstor.org/stable/42920909}.

\bibitem[\citeproctext]{ref-blackmarblepy}
Stefanini Vicente, Gabriel, and Robert Marty. 2023.
\emph{{BlackMarblePy: Georeferenced Rasters and Statistics of Nighttime
Lights from NASA Black Marble}}.
\href{https://worldbank.github.io/blackmarblepy}{Https://worldbank.github.io/blackmarblepy}.
\url{https://doi.org/10.5281/zenodo.10667907}.

\end{CSLReferences}

\end{document}
