% Pandoc/Quarto template for den-paper extension
\documentclass[
  manuscript=article,
  layout=publish,
  year=2026,
  volume=1,
]{den}

% DOI
\doi{xx.xxxxx/den.xxxx.xxxx}

% Conference (for proceedings/poster)
\conference{Conference Title}

% Dates
\received{1 April 2026}
\revised{1 May 2026}
\accepted{10 May 2026}
\published{20 May 2026}

% Editor and reviewers
\editor{Editor Name}
\reviewers{First Reviewer, Second Reviewer, Third Reviewer}

% Bibliography
\addbibresource{bibliography.bib}

% Title
\title{User Guide for the DEN Paper Template}

% Authors and affiliations
\author{First Author \orcid{0000-0000-0000-0000}}
\affiliation{Institution-1, City, Country}
\email{correspondence@email.domain}
\author{Second Author \orcid{0000-0000-0000-0000}}
\affiliation{Institution-2, City, Country}
\author{Third Author}
\affiliation{Institution-3, City, Country}

% Keywords
\keywords{keyword; keyword-two; keyword number three}

% Abbreviations

% Pandoc/Quarto compatibility macros
\providecommand{\tightlist}{%
  \setlength{\itemsep}{0pt}\setlength{\parskip}{0pt}}

% Support for graphics
\RequirePackage{graphicx}

% CSL references support

% Header includes
\makeatletter
\@ifpackageloaded{caption}{}{\usepackage{caption}}
\AtBeginDocument{%
\ifdefined\contentsname
  \renewcommand*\contentsname{Table of contents}
\else
  \newcommand\contentsname{Table of contents}
\fi
\ifdefined\listfigurename
  \renewcommand*\listfigurename{List of Figures}
\else
  \newcommand\listfigurename{List of Figures}
\fi
\ifdefined\listtablename
  \renewcommand*\listtablename{List of Tables}
\else
  \newcommand\listtablename{List of Tables}
\fi
\ifdefined\figurename
  \renewcommand*\figurename{Figure}
\else
  \newcommand\figurename{Figure}
\fi
\ifdefined\tablename
  \renewcommand*\tablename{Table}
\else
  \newcommand\tablename{Table}
\fi
}
\@ifpackageloaded{float}{}{\usepackage{float}}
\floatstyle{ruled}
\@ifundefined{c@chapter}{\newfloat{codelisting}{h}{lop}}{\newfloat{codelisting}{h}{lop}[chapter]}
\floatname{codelisting}{Listing}
\newcommand*\listoflistings{\listof{codelisting}{List of Listings}}
\makeatother
\makeatletter
\makeatother
\makeatletter
\@ifpackageloaded{caption}{}{\usepackage{caption}}
\@ifpackageloaded{subcaption}{}{\usepackage{subcaption}}
\makeatother

\begin{document}

\begin{abstract}
An abstract summarizes, in one paragraph of 300 words or fewer, the
major aspects of the entire paper. It should include: 1) the overall
purpose of the study and the research problem you investigated; 2) the
basic design of your research approach; 3) major findings as a result of
your analysis; and 4) a brief summary of your interpretations and
conclusions.
\end{abstract}

\section{User Guide for This
Template}\label{user-guide-for-this-template}

\textbf{DO NOT submit papers containing LaTeX error messages.} If you
see any error messages, please fix them before submission.

\subsection{Title}\label{title}

The title should be in Title Case. Keep it concise and informative,
ideally within 12 words. Avoid abbreviations in the title unless they
are widely recognized.

\subsection{References}\label{references}

Add your bibliography entries to \texttt{bibliography.bib} and cite them
using standard Quarto citation syntax. Here is an example citation on
diamond open access \autocite{fuchs2013diamond}. Since many of you are
using the OpenSky data, here is another example
\autocite{schafer2014bringing}.

\subsection{Footnotes}\label{footnotes}

Use footnotes sparingly. They should be used for additional information
that is not essential to the main text.\footnote{This is an example
  footnote that works.}

\subsection{Tables}\label{tables}

Use standard markdown tables or Quarto's table syntax.
Table~\ref{tbl-example} shows an example.

\begin{longtable}[]{@{}lll@{}}
\caption{Example table}\label{tbl-example}\tabularnewline
\toprule\noalign{}
\textbf{Parameter} & \textbf{Notation} & \textbf{Remarks} \\
\midrule\noalign{}
\endfirsthead
\toprule\noalign{}
\textbf{Parameter} & \textbf{Notation} & \textbf{Remarks} \\
\midrule\noalign{}
\endhead
\bottomrule\noalign{}
\endlastfoot
name & - & engine common identifier \\
manufacture & - & name of the manufacturer \\
bpr & \(\lambda\) & bypass ratio \\
pr & - & pressure ratio \\
thrust & \(T_0\) & maximum static thrust \\
\end{longtable}

\subsection{Figures}\label{figures}

Use standard Quarto figure syntax with
\texttt{!{[}caption{]}(path)\{\#fig-id\ width=X\%\}}. Place images in a
\texttt{figures/} folder.

\subsection{Equations}\label{equations}

Use standard LaTeX equation syntax. Equation~\ref{eq-cauchy} shows an
example equation.

\begin{equation}\phantomsection\label{eq-cauchy}{
\rho\frac{\mathrm{D} \mathbf{u}}{\mathrm{D} t} = - \nabla p + \nabla \cdot \boldsymbol \tau + \rho\,\mathbf{g}
}\end{equation}

\section{Sections}\label{sections}

Organize your paper using standard markdown headings.

Some standard sections are:

\begin{itemize}
\tightlist
\item
  Introduction
\item
  Methods
\item
  Results
\item
  Discussion
\item
  Conclusion
\end{itemize}

\section*{Acknowledgement}\label{acknowledgement}
\addcontentsline{toc}{section}{Acknowledgement}

Include your acknowledgements in this section.

\section*{Author Contributions}\label{author-contributions}
\addcontentsline{toc}{section}{Author Contributions}

If the paper has more than one author, the CRediT section must be
included. See example usage at \url{https://casrai.org/credit/}

\begin{itemize}
\tightlist
\item
  First Author: Conceptualization, Data Curation, Formal Analysis
\item
  Second Author: Data Curation, Writing - Original Draft
\item
  Third Author: Visualization, Investigation
\end{itemize}

\section*{Funding Statement}\label{funding-statement}
\addcontentsline{toc}{section}{Funding Statement}

When applicable, please specify the funding information for this
research.

\section*{Open Data Statement}\label{open-data-statement}
\addcontentsline{toc}{section}{Open Data Statement}

\textbf{Mandatory section!}

Include DOI and a short description of supplementary data.

\section*{Reproducibility Statement}\label{reproducibility-statement}
\addcontentsline{toc}{section}{Reproducibility Statement}

\textbf{Mandatory section!}

Information on how to reproduce this research, including access to: 1)
source code related to the research, 2) source code for the figures, 3)
source code/data for the tables when applicable.

\printbibliography

\end{document}
